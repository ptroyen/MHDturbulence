\documentclass{beamer}
%\usetheme{Berlin}
\usecolortheme{seahorse}

\usepackage[utf8]{inputenc}
\usepackage{default}
\usepackage{amsmath}
\usepackage{bm}
\usepackage{xmpmulti}
%\usepackage[backend=bibtex]{biblatex}
%\bibliography{DNS_in_MHD}

\title{Use of Direct Numerical Simulations for Studies on Magnetohydrodynamics}
\author{Andr\'es Cathey}
\institute{The University of Edinburgh}
\date{\today}

\begin{document}

\frame{\titlepage}


\begin{frame}
\frametitle{Overview}
\tableofcontents
\end{frame}

\AtBeginSection[]
{
  \begin{frame}
    \frametitle{Overview}
    \tableofcontents[currentsection]
  \end{frame}
}

\section{Direct Numerical Simulations (DNS)}

\subsection{Foundations}

\begin{frame}
 \frametitle{Direct Numerical Simulations (DNS)} %unrivalled in accuracy and in the level of description provided
 \framesubtitle{Foundations}
 
 \begin{itemize}
  \item Subsection of Computational Fluid Dynamics.
  %\begin{itemize}
   %\item Solve PDEs that govern the motion of real fluids.
  %\end{itemize}
  \item<2-> Navier-Stokes equation for an incompressible fluid ($\bm \nabla \cdot \bm U = 0$): \begin{equation}
                                \frac{\partial \bm U}{\partial t} + (\bm U \cdot \bm \nabla) \bm U = - \frac{1}{\rho} \bm \nabla P + \nu \nabla ^ 2 \bm U + \bm F
                               \end{equation}
  \item<3-> Reynolds number: \begin{equation}
                              Re = \frac{\text{inertial forces}}{\text{viscous forces}} = \frac{u L}{\nu}
                             \end{equation} % onset of turbulence

  \item<4-> Non-linear term allows energy transfer from large to and from small scales.
  %That plus viscous term are key in the simulation of turbulence. 
  %the cost is extremely high; and the computer requirements increase so rapidly with Reynolds number that the applicability of the approach is limited to flows of low or moderate Reynolds numbers.
 \end{itemize}
  
\end{frame}

\begin{frame}
 \frametitle{Direct Numerical Simulations (DNS)}
 \framesubtitle{Foundations}
 
 \begin{itemize}
  \item Pioneered by Orszag and Patterson (1972) and Rogallo (1981).
  \item<2-> Pseudo-spectral method.%, with velocity field as (finite) Fourier series. % (i.e. Fourier Space).
  \item<3-> High computational cost. \begin{itemize}
                                      \item<4-> $N^3$ grid points in a cubical domain with sides of length $L=2\pi$. 
                                      \item<5-> Can be related to the Reynolds number (Landau and Liftshitz, 1959).
                                      \item<6-> Has to represent large-scales.
                                      \item<7-> Grid spacing $\Delta x$ small enough to resolve dissipative scales.
                                     \end{itemize}

  % N^3 wavenumber are analysed. 
 \end{itemize}
\end{frame}

\subsection{Method}

\begin{frame}
 \frametitle{Direct Numerical Simulations (DNS)}
 \framesubtitle{Method}
 
 %isotropic velocity field with zero mean: U = u (u is the variation of the velocity field)
 
 Start with a reformulated Navier Stokes equation in Fourier space. % Solenoidal NSE: system with boundary is zero at infinity and there are no mean velocities or externally applied pressure gradients.
 
 \begin{equation}
  \left( \frac{\partial}{\partial t} - \nu \nabla^2 \right) u_\alpha (\bm k, t) = M_{\alpha \beta \gamma}(\bm k) \left[ \sum_{\bm j + \bm l = \bm k} u_\beta (\bm j, t) u_\gamma (\bm l, t) \right]
 \end{equation}%M is a symmetric operator
 
 % I won't go into this, but it arises after introducing the zero boundary conditions at infinity and expressing the pressure (with the use of Green's functions) as a function of the velocity field. 
 A discretisation in time and wavenumber is needed to obtain a numerical solution. % this means bounding the summation and it generates aliasing errors.
 
 \begin{equation}
  k_\alpha = \frac{2 \pi}{L} n_\alpha \qquad \qquad \qquad n_\alpha = 0, 1, \ldots, N - 1
 \end{equation}
 % clearly the size of the simulation depends on the number of wavenumbers calculated. K wavenumbers are calculated.
 
 %This is equivalent to analysing a cubical grid with \Delta x = L / K = \pi / k_{max}
 
 Non-linear term:
 
 \begin{equation}
  A_{\beta \gamma} (\bm k, t) = \sum_{\bm j + \bm l = \bm k} u_\beta (\bm j, t) u_\gamma (\bm l, t)
 \end{equation}
 
\end{frame}

\begin{frame}
 \frametitle{Direct Numerical Simulations (DNS)}
 \framesubtitle{Method}
 
 The non-linear term is computationally costly. The convolution theorem can help with that.
 
 \begin{enumerate}
  \item Perform Fast Fourier Transform (FFT) on velocity fields ($u_\alpha (\bm x, t)$). % numerically (and in C++) this is done with FFTW. cost ~ O(N ln(N))
  \item<2-> Convolve the transformed fields. \begin{equation}
                                              A_{\beta \gamma} (\bm x, t) = u_\beta (\bm x, t) u_\gamma (\bm x, t)
                                             \end{equation}
  \item<3-> Transform $A_{\beta \gamma} (\bm x, t)$ to $A_{\beta \gamma} (\bm k, t)$ to obtain the l.h.s term in the solenoidal Navier Stokes equation.
  \item<4-> Use a suitable finite-differences method for evolving the velocity field in time~\cite{mccomb1990physics}. 
 \end{enumerate}

\end{frame}

\subsection{Why Pseudo-Spectral?}

\begin{frame}
 \frametitle{Direct Numerical Simulations (DNS)}
 \framesubtitle{Why Pseudo-Spectral?}
 
 So... Why pseudo-spectral? \pause
 
 Performing FFT can be a costly endeavor. ($N \log_2 N$) \pause
 \vspace{1cm}
 
 {\color{red}Decrease in computational cost!} \pause
 
 Arithmetic operations required for calculating the non-linear term:
 \begin{itemize}
  \item Before FFT: $\mathcal O (N^6)$.
  \item<5-> After FFT: $\mathcal O (N^3 \log_2 N)$.
 \end{itemize}

\end{frame}

\begin{frame}
 \frametitle{Direct Numerical Simulations (DNS)}
 \framesubtitle{Why Pseudo-Spectral?}
 
 Considerations for DNS code~\cite{yoffe2013investigation}:
 
 \begin{itemize}
  \item<2-> Time-stepping strategy.
  \item<3-> Aliasing errors.
  \item<4-> Implementation of parallelisation.
  \item<5-> Benchmarking.
 \end{itemize}


\end{frame}


\section{Magnetohydrodynamics with DNS}

\subsection{Framework}

\begin{frame}
 \frametitle{Magnetohydrodynamics with DNS}
 \framesubtitle{Framework}
 
 Magnetohydrodynamics (MHD) study electrically conducting fluids. \pause
 
 Equations of fluid motion are obtained by combining Navier Stokes equations with Maxwell's equations.
 
 \begin{equation}
  \frac{\partial \bm u}{\partial t} + (\bm u \cdot \bm \nabla) \bm u = - \frac{1}{\rho} \bm \nabla P + \frac{1}{\rho} (\bm \nabla \times \bm b) \times \bm b + \nu \nabla ^ 2 \bm u + \bm f_u
 \end{equation} % (nabla cross b) cross b == (b dot nabla) b. that is the lorentz force.
 
 \begin{equation}
  \frac{\partial \bm b}{\partial t} + (\bm u \cdot \bm \nabla) \bm b = (\bm b \cdot \bm \nabla) \bm u + \eta \nabla ^ 2 \bm b + \bm f_b
 \end{equation} \pause
 
 \begin{equation}
  Re = \frac{u L}{\nu} \qquad \qquad \qquad \qquad Re_m = \frac{u L}{\eta}
 \end{equation}


\end{frame}

\subsection{Magnetic Prandtl Number}

\begin{frame}
 \frametitle{Magnetohydrodynamics with DNS}
 \framesubtitle{Magnetic Prandtl Number}
 
 \begin{align}
 Re = \frac{\text{inertial forces}}{\text{viscous forces}} =  \frac{u\,L}{\nu} \nonumber \qquad
 Re_M = \frac{\text{inertial forces}}{\text{diffusive forces}} =  \frac{u\,L}{\eta} \nonumber
 \end{align}\pause
 
 \begin{equation}
  Pr_M = \frac{Re_M}{Re} =  \frac{\nu}{\eta}
 \end{equation} \pause
 
 $Pr_m < 1$ when most energy is dissipated through the magnetic channel.
 
 $Pr_m > 1$ when most energy is dissipated through the kinetic (viscous) channel.
 
\end{frame}

\begin{frame}
 \frametitle{Magnetohydrodynamics with DNS}
 \framesubtitle{Magnetic Prandtl Number}
 
 \begin{figure}[t]
  \includegraphics[width=8cm]{img/PrM_spectrum}
  \caption{Map of ``typical'' objects in the plane $(Re,Re_M)$. Yellow dashed lines are $Pr_M$ isolines. \cite{plunian2013shell}.}
  \centering
 \end{figure}
 
 %Earth dynamo: $Pr_M \sim \mathcal{O}(10^{-7})$.
 
 %Inside the Sun: $Pr_M \sim \mathcal{O}(10^{-3})$.
 
 %Thin accretion disks: $Pr_M \sim \mathcal{O}(10^0)$.
 
 %Solar corona: $Pr_M \sim \mathcal{O}(10^4) - \mathcal{O}(10^8)$. \cite{hood2011solar}
 
\end{frame}

\section{DNS at Work}

\subsection{Non-unity Magnetic Prandtl Number}

\begin{frame}
 \frametitle{DNS at Work}
 \framesubtitle{Non-unity Magnetic Prandtl Number}
 
 A well known research on non-unity magnetic Prandtl numbers is:\vspace{1cm}
 
 G. Sahoo, P. Perlekar, and R. Pandit, “Systematics of the magnetic-prandtl-number dependence of homogeneous, isotropic magnetohydrodynamic turbulence”.\vspace{1cm}\pause
 
 Range of achieved Magnetic Prandtl Numbers: $[0.01:10]$.
 
\end{frame}

\begin{frame}
 \frametitle{DNS at Work}
 \framesubtitle{Non-unity Magnetic Prandtl Number}
 
 Sahoo et. al. use a DNS method with:
 
 \begin{itemize}
  \item Time-stepping strategy is the Adams-Bashforth scheme. %this uses an interpolation method. dt satisfies Courant–Friedrichs–Lewy condition (convergence condition).
  \item<2-> $N = 512$ and $N = 1024$.
  \item<3-> Study decaying and statistically steady turbulence. 
  \item<4-> $2/3-$rule as a dealiasing method. 
 \end{itemize}

 
 %$f_u, f_b = 0$ for decaying turbulence.
 
 %Steady forcing for statistically steady simulation (think of $\partial / \partial t \rightarrow 0 $).
 
\end{frame}

\begin{frame}
 \frametitle{DNS at Work}
 \framesubtitle{Non-unity Magnetic Prandtl Number}
 
 \begin{itemize}
  \item Time-stepping strategy is the Adams-Bashforth scheme. %this uses an interpolation method. dt satisfies Courant–Friedrichs–Lewy condition (convergence condition).
  \item<2-> $N = 512$ and $N = 1024$.
  \item<3-> Study decaying and statistically steady turbulence. 
  \item<4-> $2/3-$rule as a dealiasing method. 
 \end{itemize}

 
 %$f_u, f_b = 0$ for decaying turbulence.
 
 %Steady forcing for statistically steady simulation (think of $\partial / \partial t \rightarrow 0 $).
 
\end{frame}

\subsection{Results}

%\begin{frame}
 %\frametitle{DNS at Work}
 %\framesubtitle{Results}
 
 %$Pr_M = 0.1$ \pause
 
 %\begin{figure}[ht]
        %\begin{minipage}[b]{0.45\linewidth}
         %   \centering
          %  \includegraphics[width=\textwidth]{img/energy_a1}
           % \caption{Time evolution of total (red), kinetic (green), and magnetic (blue) energies. \cite{sahoo2011systematics}.}
   %         \label{fig:a}
    %    \end{minipage}
     %   \hspace{0.05cm}
      %  \begin{minipage}[b]{0.46\linewidth}
       %     \centering
        %    \includegraphics[width=\textwidth]{img/energy_a2}
         %   \caption{Time evolution of total, kinetic, and magnetic energy dissipation. \cite{sahoo2011systematics}.}
          %  \label{fig:b}
        %\end{minipage}
    %\end{figure}
 
%\end{frame}

%\begin{frame}
 %\frametitle{DNS at Work}
 %\framesubtitle{Results}
 
 %$Pr_M = 0.1$
 
 %\begin{figure}
 % \includegraphics[width=0.6\textwidth]{img/energy_a3}
  %\caption{Magnetic-to-Kinetic energy ratio for $Pr_M = 0.1$\cite{sahoo2011systematics}.}
  %\centering
 %\end{figure}
 
%\end{frame}

%\begin{frame}
 %\frametitle{DNS at Work}
 %\framesubtitle{Results}
 
 %$Pr_M = 10.0$
 
  %   \begin{figure}[ht]
   %     \begin{minipage}[b]{0.45\linewidth}
    %        \centering
     %       \includegraphics[width=\textwidth]{img/energy_e1}
      %      \caption{Time evolution of total (red), kinetic (green), and magnetic (blue) energies. \cite{sahoo2011systematics}.}
       %     \label{fig:a}
       % \end{minipage}
       % \hspace{0.05cm}
       % \begin{minipage}[b]{0.46\linewidth}
       %     \centering
        %    \includegraphics[width=\textwidth]{img/energy_e2}
         %   \caption{Time evolution of total, kinetic, and magnetic energy dissipation. \cite{sahoo2011systematics}.}
         %   \label{fig:b}
        %\end{minipage}
    %\end{figure}
 
%\end{frame}

%\begin{frame}
 %\frametitle{DNS at Work}
 %\framesubtitle{Results}
 
 %$Pr_M = 10.0$
 
 %\begin{figure}
  %\includegraphics[width=0.6\textwidth]{img/energy_e3}
  %\caption{Magnetic-to-Kinetic energy ratio for $Pr_M = 10.0$\cite{sahoo2011systematics}.}
  %\centering
 %\end{figure}
 
%\end{frame}

\begin{frame}
 \frametitle{DNS at Work}
 \framesubtitle{Results}
 
 Study decaying simulations with varying magnetic Prandtl Number. \pause
 
 \begin{equation}
  \frac{\partial \bm u}{\partial t} + (\bm u \cdot \bm \nabla) \bm u = - \frac{1}{\rho} \bm \nabla P + \frac{1}{\rho} (\bm \nabla \times \bm b) \times \bm b + \nu \nabla ^ 2 \bm u + \bm f_u
 \end{equation} % (nabla cross b) cross b == (b dot nabla) b. that is the lorentz force.
 
 \begin{equation}
  \frac{\partial \bm b}{\partial t} + (\bm u \cdot \bm \nabla) \bm b = (\bm b \cdot \bm \nabla) \bm u + \eta \nabla ^ 2 \bm b + \bm f_b
 \end{equation}
 
 \begin{equation}
  \bm f_u = \bm 0 \qquad \qquad \qquad \bm f_b = \bm 0
 \end{equation}

 
\end{frame}


%\begin{frame}
 %\frametitle{DNS at Work}
 %\framesubtitle{Results}
 
 %Runs (f-h) have higher Reynolds numbers compared to their (a-e) equivalents.
 
 %\begin{figure}
  %\includegraphics[width=0.95\textwidth]{img/energies_prm}
  %\caption{Total (red), kinetic (green), and magnetic (blue) energies for a:~$Pr_M = 0.1$, b:~$Pr_M = 0.5$, c:~$Pr_M = 1.0$, d:~$Pr_M = 5.0$, e:~$Pr_M = 10.0$, f:~$Pr_M = 1.0$, g:~$Pr_M = 5.0$, h:~$Pr_M = 10.0$ \cite{sahoo2011systematics}.}
  %\centering
 %\end{figure}
 
%\end{frame}

\begin{frame}
 \frametitle{DNS at Work}
 \framesubtitle{Results}
 
 Runs (f-h) have higher Reynolds numbers compared to their (a-e) equivalents.
 
 \begin{figure}
  \includegraphics[width=0.95\textwidth]{img/energy_dissipation_prm.png}
  \caption{Total (red), kinetic (green), and magnetic (blue) energy dissipation ratios for a:~$Pr_M = 0.1$, b:~$Pr_M = 0.5$, c:~$Pr_M = 1.0$, d:~$Pr_M = 5.0$, e:~$Pr_M = 10.0$, f:~$Pr_M = 1.0$, g:~$Pr_M = 5.0$, h:~$Pr_M = 10.0$ \cite{sahoo2011systematics}.}
  \centering
 \end{figure}
 
\end{frame}

%\begin{frame}
 %\frametitle{DNS at Work}
 %\framesubtitle{Results}
 
 %Runs (f-h) have higher Reynolds numbers compared to their (a-e) equivalents.
 
 %\begin{figure}
  %\includegraphics[width=0.95\textwidth]{img/energy_relation_prm}
  %\caption{Total (red), kinetic (green), and magnetic (blue) energies for a:~$Pr_M = 0.1$, b:~$Pr_M = 0.5$, c:~$Pr_M = 1.0$, d:~$Pr_M = 5.0$, e:~$Pr_M = 10.0$, f:~$Pr_M = 1.0$, g:~$Pr_M = 5.0$, h:~$Pr_M = 10.0$ \cite{sahoo2011systematics}.}
  %\centering
 %\end{figure}
 
%\end{frame}



\section{Conclusions}

\begin{frame}
 \frametitle{Conclusions}
 
 \begin{enumerate}
  \item<1-> Direct Numerical Simulations look in a large spectral range.
  \item<2-> To avoid extremely high computational complexity ($\mathcal{O}(N^6)$), convolution theorem and FFTs can help.
  \item<3-> Increases in computing power will allow broadening $Pr_M$ ranges under study.
 \end{enumerate}
 
\end{frame}

\begin{frame}
\frametitle{References}

\setbeamerfont{bibliography item}{size=\footnotesize}
\setbeamerfont{bibliography entry author}{size=\footnotesize}
\setbeamerfont{bibliography entry title}{size=\footnotesize}
\setbeamerfont{bibliography entry location}{size=\footnotesize}
\setbeamerfont{bibliography entry note}{size=\footnotesize}
\setbeamertemplate{bibliography item}[text]

\bibliographystyle{ieeetr}
\bibliography{DNS_in_MHD}

\end{frame}

\begin{frame}
 \frametitle{Aliasing Errors}
  
 \begin{figure}[t]
  \includegraphics[width=0.8\textwidth]{img/aliasing}
  \caption{Aliasing exemplified.}
  \centering
 \end{figure}
  
\end{frame}

\begin{frame}
 \frametitle{Run Information}
  
 \begin{figure}[t]
  \includegraphics[width=0.95\textwidth]{img/run_info}
  \caption{$k_{max}$ is the magnitude of the largest-magnitude wave vectors resolved in these DNS studies which use the $2/3$ dealiasing rule; $k_{max} \approx 170.67$ and $341.33$ for $N = 512$ and $1024$, respectively~\cite{sahoo2011systematics}.}
  \centering
 \end{figure}
  
\end{frame}

\begin{frame}
 \frametitle{Shell Models}
 \framesubtitle{Energy profiles with shell model}
  
   \begin{figure}[t]
  \includegraphics[width=5cm]{img/E_B_profile}
  \caption{Compensated time-averaged kinetic and magnetic energy spectra for
shell models at three values of $Pr_M$ \cite{brandenburg2014magnetic}.}
  \centering
 \end{figure}
  
\end{frame}


\end{document}
